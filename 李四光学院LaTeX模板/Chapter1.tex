

% %%------------------------document环境开始------------------------%%

%%-----------------------封面--------------------%%
\cover
\thispagestyle{empty} % 首页不显示页码

%%--------------------------目录页------------------------%%
\newpage
\tableofcontents
 \thispagestyle{empty} % 目录不显示页码

%%------------------------正文页从这里开始-------------------%
\newpage
\setcounter{page}{1} % 让页码从正文开始编号


\iffalse
\begin{itemize}
    \item \texttt{main.tex} 主文件
    \item \texttt{reference.bib} 参考文献,使用bibtex
    \item \texttt{CUGReport.sty} 文档格式控制,包括一些基础的设置,如页眉、标题、学院、学号、姓名等
    \item \texttt{figures} 放置图片的文件夹
\end{itemize}
\fi
\section{单像空间后方交会}

\subsection{实验原理与流程}

\subsubsection{实验原理}
\textbf{此为公式与图片输入方法示意:}

\begin{figure}[!htbp]
	\centering
	\includegraphics[width=0.7\linewidth]{figures/cug.jpg}
	\caption{单像空间后方交会原理图}
	\label{fig:原理}
\end{figure}
\FloatBarrier

详细流程:
	基本关系式为共线条件方程式:
    \begin{align*}
        x &= -f \frac{a_1(X - X_S) + b_1(Y - Y_S) + c_1(Z - Z_S)}{a_3(X - X_S) + b_3(Y - Y_S) + c_3(Z - Z_S)} = -f \frac{\overline{X}}{Z} \\
        y &= -f \frac{a_2(X - X_S) + b_2(Y - Y_S) + c_2(Z - Z_S)}{a_3(X - X_S) + b_3(Y - Y_S) + c_3(Z - Z_S)} = -f \frac{\overline{Y}}{Z}
    \end{align*}
    其中,$f$为焦距,$a_1,b_1,c_1,a_2,b_2,c_2,a_3,b_3,c_3$为外方位元素,$X_S,Y_S,Z_S$为相对于像片坐标系的相对坐标,$X,Y,Z$为地面点的地理坐标,$x,y$为像点的坐标。
    通过以上方程,可以得到三个方程,通过解方程组,可以求解出外方位元素。

    用矩阵形式表示:
    \begin{gather*}
        V = AX - l \\
        V = [v_x, \quad v_y]^\text{T} \\
        A = \begin{bmatrix}
            a_{11} & a_{12} & a_{13} & a_{14} & a_{15} & a_{16} \\
            a_{21} & a_{22} & a_{23} & a_{24} & a_{25} & a_{26}
        \end{bmatrix} \\
        X = \left[ dX_S \quad dY_S \quad dZ_S \quad d\varphi \quad d\omega \quad d\kappa \right]^T \\
        l = \left[ l_x \quad l_y \right]^T
    \end{gather*}
    
\subsection{实习数据与代码部分}

\subsubsection{实习数据}    

请输入文本

\subsubsection{实习要求}
XXX请输入文本
\subsubsection{实习代码}
详细代码请见报告末尾附录A部分的文件列表示意与附录B的代码部分,在行文中不再加以赘述。

\subsection{实习结果与数据分析}
\subsubsection{实习结果}
XXX请输入文本


\subsubsection{成果分析}
XXX请输入文本
\subsection{单次实习小结}
XXX请输入文本
