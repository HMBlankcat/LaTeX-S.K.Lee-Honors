%!TeX program = xelatex
\documentclass[12pt,hyperref,a4paper,UTF8]{ctexart}
\usepackage{CUGReport}
\usepackage{listings}
\usepackage{xcolor}
\usepackage{fontspec}
\usepackage{setspace}
\usepackage{fancyhdr}
\usepackage[section]{placeins}
\setstretch{1.5} % 设置全局行距为1.5倍

\usepackage{enumitem} % 载入enumitem包以便自定义列表环境
\setlist[itemize]{itemsep=0pt, parsep=0pt} % 设置itemize环境的项目间距和段落间距

\setmainfont{Times New Roman} % 英文正文为Times New Roman
\fancyhead[C]{中国地质大学(武汉)\ \ 李四光学院 \ \ Name}
%字号设置
\newcommand{\xiaochuhao}{\fontsize{36pt}{\baselineskip}\selectfont}
\newcommand{\erhao}{\fontsize{21pt}{\baselineskip}\selectfont}
\newcommand{\xiaoerhao}{\fontsize{18pt}{\baselineskip}\selectfont}
\newcommand{\sanhao}{\fontsize{15.75pt}{\baselineskip}\selectfont}
\newcommand{\sihao}{\fontsize{14pt}{18pt}\selectfont}
\newcommand{\xiaosihao}{\fontsize{12pt}{18pt}\selectfont}
%\newcommand{\wuhao}{\fontsize{10.5pt}{18pt}\selectfont}


%封面页设置
{   
    %标题
    \title{ 
        \heiti \xiaochuhao \textbf{{Course name}} \par
        \heiti \xiaochuhao \textbf{{课程实习报告}} \par
        %\vspace{1cm} 
       % \heiti \Large {\underline{XXXXXX进展调研}}    
        %\vspace{1cm}
    }

    \author{
        \vspace{0.5cm}
        \kaishu\Large 学生姓名\ \dlmu[9cm]{Name} \qquad \\ %姓名 
        \vspace{0.5cm}
        \kaishu\Large 学\hspace{2em}院\ \dlmu[9cm]{李四光学院} \qquad \\ %学院
        \vspace{0.5cm}
        \kaishu\Large 班\hspace{2em}级\ \dlmu[9cm]{201226} \\ %班级
        \vspace{0.5cm}
        \kaishu\Large 学\hspace{2em}号\ \dlmu[9cm]{2022100XXXX} \\ %学号
        \vspace{0.5cm}
        %\vspace{0.5cm}
        \kaishu\Large 授课教师\ \dlmu[9cm]{Teacher Name} \qquad  \\ 
        \vspace{1cm}
      	%\vspace{0.25cm} 
    }
    \date{\today} % 默认为今天的日期,可以注释掉不显示日期
}
% %%------------------------document环境开始------------------------%%


\begin{document}
	

% %%------------------------document环境开始------------------------%%

%%-----------------------封面--------------------%%
\cover
\thispagestyle{empty} % 首页不显示页码

%%--------------------------目录页------------------------%%
\newpage
\tableofcontents
 \thispagestyle{empty} % 目录不显示页码

%%------------------------正文页从这里开始-------------------%
\newpage
\setcounter{page}{1} % 让页码从正文开始编号


\iffalse
\begin{itemize}
    \item \texttt{main.tex} 主文件
    \item \texttt{reference.bib} 参考文献,使用bibtex
    \item \texttt{CUGReport.sty} 文档格式控制,包括一些基础的设置,如页眉、标题、学院、学号、姓名等
    \item \texttt{figures} 放置图片的文件夹
\end{itemize}
\fi
\section{单像空间后方交会}

\subsection{实验原理与流程}

\subsubsection{实验原理}
\textbf{此为公式与图片输入方法示意:}

\begin{figure}[!htbp]
	\centering
	\includegraphics[width=0.7\linewidth]{figures/cug.jpg}
	\caption{单像空间后方交会原理图}
	\label{fig:原理}
\end{figure}
\FloatBarrier

详细流程:
	基本关系式为共线条件方程式:
    \begin{align*}
        x &= -f \frac{a_1(X - X_S) + b_1(Y - Y_S) + c_1(Z - Z_S)}{a_3(X - X_S) + b_3(Y - Y_S) + c_3(Z - Z_S)} = -f \frac{\overline{X}}{Z} \\
        y &= -f \frac{a_2(X - X_S) + b_2(Y - Y_S) + c_2(Z - Z_S)}{a_3(X - X_S) + b_3(Y - Y_S) + c_3(Z - Z_S)} = -f \frac{\overline{Y}}{Z}
    \end{align*}
    其中,$f$为焦距,$a_1,b_1,c_1,a_2,b_2,c_2,a_3,b_3,c_3$为外方位元素,$X_S,Y_S,Z_S$为相对于像片坐标系的相对坐标,$X,Y,Z$为地面点的地理坐标,$x,y$为像点的坐标。
    通过以上方程,可以得到三个方程,通过解方程组,可以求解出外方位元素。

    用矩阵形式表示:
    \begin{gather*}
        V = AX - l \\
        V = [v_x, \quad v_y]^\text{T} \\
        A = \begin{bmatrix}
            a_{11} & a_{12} & a_{13} & a_{14} & a_{15} & a_{16} \\
            a_{21} & a_{22} & a_{23} & a_{24} & a_{25} & a_{26}
        \end{bmatrix} \\
        X = \left[ dX_S \quad dY_S \quad dZ_S \quad d\varphi \quad d\omega \quad d\kappa \right]^T \\
        l = \left[ l_x \quad l_y \right]^T
    \end{gather*}
    
\subsection{实习数据与代码部分}

\subsubsection{实习数据}    

请输入文本

\subsubsection{实习要求}
XXX请输入文本
\subsubsection{实习代码}
详细代码请见报告末尾附录A部分的文件列表示意与附录B的代码部分,在行文中不再加以赘述。

\subsection{实习结果与数据分析}
\subsubsection{实习结果}
XXX请输入文本


\subsubsection{成果分析}
XXX请输入文本
\subsection{单次实习小结}
XXX请输入文本

    % 采用分割的方式,每个章节一个文件,只需在文件夹下创建对应文件名的tex文件并include即可
    
\section{写在最后}
\subsection{发布地址}
代码将在本人的Github上发布并进行存档。
\begin{itemize}
    \item Github: \url{https://github.com/HMBlankcat}
\end{itemize}

\section{实习体会}

请输入文本


\begin{appendices}

\section{文件列表}

\begin{table}[h]  
	\centering  
	\caption{文件列表}  
	\renewcommand{\arraystretch}{1.25} % 增加行间距  
	\begin{tabular}{c@{\hspace{20pt}}c} % 增加列间距   
		\toprule  
		文件名   & 功能描述 \\
		\midrule  
		单像空间后方交会.py & 单像空间后方交会程序代码 \\
		% 同样的方式可继续增添
		\toprule  
	\end{tabular}  
	\label{tab:文件列表}  
\end{table}  
	
	\section{代码}
	\noindent 单像空间后方交会.py
	\lstinputlisting[language=python]{code/单像空间后方交会.py}

\end{appendices}

%\end{document}
\end{document}
